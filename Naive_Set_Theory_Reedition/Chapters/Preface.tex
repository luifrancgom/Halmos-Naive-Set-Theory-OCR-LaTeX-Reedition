%--- PREFACE PAGES 10-11
%--- MODIFIED LAST IN 29/05/2023 (DD/MM/YYYY)


Every mathematician agrees that every mathematician must know some set theory; the disagreement begins in trying to decide how much is some. This book contains my answer to that question. The purpose of the book is to tell the beginning student of advanced mathematics the basic set-theoretic facts of life, and to do so with the minimum of philosophical discourse and logical  formalism. The point of view throughout is that prospective mathematician anxious to study groups, or integrals, or manifolds. From this point of view the concepts and methods of this book are merely some of the standard mathematical tools; the expert specialist will find nothing new here.

Scholarly bibliographical credits and references are out of place in a purely expository book such as this one. The student who gets interested in set theory for its own sake should know, bowever, that there is much more to the subject than there is in this book. One of the most beautiful sources of set-theoretic wisdom is still Hausdorff's \textit{Set theory}. A recent and highly readable addition to the literature, with an extensive and up-to-date bibliography, is \textit{Axiomatic set theory} by Suppes. 

In set theory "naive" and "axiomatic" are contrasting words. The present treatment might best be described as axiomatic set theory from the naive point of view. It is axiomatic in that some axioms for set theory are stated and used as the basis of all subsequent proofs. It is naive in that the language and notation are those of ordinary informal (but formalizable) mathematics. A more important way in which the naive point view predominates is that set theory is regarded as a body of facts, of which the axioms are a brief and convenient summary; in the orthodox axiomatic view the logical relations among various axioms are the central objects of study. Analogously, a study of geometry might be regarded purely naive if it proceeded on the paper-folding kind of intuition alone; the other extreme, the purely axiomatic one, is the one in which axioms for the various non-Euclidean geometries are studied with the same amount of attention as Euclid's. The analogue of the point of view of this book is the study of just one sane set of axioms with the intention of describing Euclidean geometry only.

Instead of \textit{Naive set theory} a more honest title for the book would have been \textit{An outline of the elements of naive set theory}. "Elements" would warn the reader that not everything is here; "outline" would warn him that even what is here needs filling in. The style is usually informal to the point of conversational. There are very few displayed theorems; most of the facts are just stated and followed by a sketch of a proof, very much as they might be in a general descriptive lecture. There are only a few exercises, officially so labelled, but, in fact, most of the book is nothing but a long chain of exercises with hints. The reader should continually ask himself whether he knows how to jump from one hint to the next, and, accordingly, he should not be discouraged if he finds that his reading rate is considerably slower than normal. 

This is not to say that the contents of this book are unusually difficult or profound. What is true is that the concepts are very general and very abstract, and that, therefore, they may take some getting used to. It is a mathematical truism, however, that the more generally a theorem applies, the less deep it is. The student's task in learning set theory is to steep himself in unfamiliar but essentially shallow generalities till they become so familiar that they can be used with almost no conscious effort. In other words, general set theory is pretty trivial stuff really, but, if you want to be a mathematician, you need some, and here it is; read it, absorb it, and forget it.

\begin{flushright}
P. R. H.
\end{flushright} 