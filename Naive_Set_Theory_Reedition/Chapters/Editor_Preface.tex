As the book title says, this is the famous set theory book \textit{Naive Set Theory} by Paul Richard Halmos, first published in 1960 by D. Van Nostrand Company, INC., part of a series called \textit{ The University Series in Undergraduate Mathematics}.

What the title doesn't say is that this version is an independent re-edition. The original work is currently public domain in \href{https://babel.hathitrust.org}{\color{blue}{Hathi Trust Digital Library}} — the reader probably found (or could find) the original digitized book on Google by just searching for its title. This version was written in LaTeX and released on July 14, 2023, available for free to download on my \href{https://github.com/matheusgirola/Halmos-Naive-Set-Theory-OCR-LaTeX-Reedition}{\color{blue}{Github repository}}.

Even though the book was freely available online, there are two reasons for this project. First, the book in its digitized state is perfectly readable, but it doesn't allow searching words with \textit{Ctrl + F} and doesn't have a interactable summary with it. The second reason is purely personal, I have a passion and gratitude for this book and, while I want to learn OCR, I decided to re-edit it as a homage.

Some notes on this re-edition are necessary. The book page format is B5 paper with font size 12pt. The margins of the book should be perfectly suitable for printing. The mainly differences with the original editions are the cover and the chapters title page designs. The mathematical symbol which denotes \textit{set inclusion} in the original is ($\epsilon$), but I opted to use ($\in$) since it's used regularly nowadays for this case. Besides this, I didn't change anything from the text. Therefore, any mistakes — which I hope are non-existent or, at least, few —  are solely mine, and if someone finds any please contact me via  \href{mailto:matheusgirola@gmail.com}{\color{blue}{e-mail}}.

As mentioned, the original book is public domain and, so, freely available in the internet. Therefore, the resulting re-edition of the book at the end of this project has no lucrative ends by any means. This re-edition cannot be used for any commercial purposes.

I thought about writting a short story about Paul R. Halmos, since it's common for books to do this specially after the author has deceased. However, I couldn't do a better job than someone just searching on Google and/or Wikipedia. So, for now, I will just say that this book has a special place in my heart. It was one of the first works that introduced and helped me push througth writting proofs. And at the end, I fell in love not only with it, but with mathematics overall. I hope that anybody that found this version can have the same outcome as I did. Now read it, absorb it and forget it.

\begin{flushright}
Matheus Girola Macedo Barbosa - 14/07/2023
\end{flushright}
